\documentclass{cce2014-meetings}
\svnInfo $Id: minutes_week03.tex 7640 2024-02-21 08:19:14Z jbri2 $

% Meeting details
\title{Meeting Minutes -- Group 4}
\author{Location: 0 B 6}
\date{28 February 2024, 16:30--17:00}

\begin{document}

\maketitle

\section*{Present}
% Members present at the meeting
Alessandro Parrella,
Gabriel Vella,
Jonas Rousseau-Morvan,
Keith Farrugia

\section*{Discussion}

\begin{enumerate}

      % There should be an item corresponding to each of the agenda items for this
      % meeting.

      \item Minutes from the previous meeting were read and approved.

      \item No matters arose from previous meetings

      \item Progress report from group members
            \begin{enumerate}
                  \item Jonas reported that the buffer size could be calculated from the
                        smallest period of the wave, as well as taking into account the
                        sampling frequency, which can either be calculated from the expected
                        wave's frequency or the suggested frequency in the lab code. $x$ is
                        the sampling frequency and $f$ is the lowest frequency of the DTMF
                        specification. Then our buffer size is $\frac{x}{f}$ because $[x]$ is
                        in bits per second and $[f]$ is in periods per second. The $[]$
                        notation is used to denote the variable's dimension.
                  \item Keith reported that the microcontroller itself has analog pins,
                        which allow for analog to digital conversion. A circuit is needed
                        to translate from line input to ADC, but this can be taken from the
                        handbook. Port 23 is the ADC port used in the lab. There may also be
                        other ports accessible for the same purpose.
                  \item Alessandro reported that the implementation of the Fourier Transform
                        must handle raw arrays and complex numbers manually, without the use
                        of any external libraries, thus meaning that they must be coded from
                        scratch. The implementation must be basic and resource-efficient. Not
                        many sources returned the amplitude apart from the frequencies.
                  \item Gabriel reported that the board has a connector named J11 to connect
                        with the character LCD. The board is set up to work in a 4-bit mode.
                        The pins for the master in slave out, spi clock, reset, register
                        select, slaved select and backlight control.
            \end{enumerate}

      \item Targets set for the week
            \begin{enumerate}
                  \item Share the research we have done with the group and see how it links together.
                  \item Start working on the design brief.
            \end{enumerate}
            
      \item No other items were discussed
\end{enumerate}

\section*{Actions}

\begin{enumerate}

      \item Write the ADC sampling subsection of the Design Brief and begin
            research on the threading \& interrupts
            \begin{flushright}
                  Assigned to: \textbf{Keith Farrugia} \\
                  Deadline: \textbf{next meeting}
            \end{flushright}

      \item Write the ADC Buffer Size subsection of the Design Brief and begin
            research on the threading \& interrupts
            \begin{flushright}
                  Assigned to: \textbf{Jonas Rousseau-Morvan} \\
                  Deadline: \textbf{next meeting}
            \end{flushright}

      \item Write the Fourier transform subsection of the Design Brief
            \begin{flushright}
                  Assigned to: \textbf{Alessandro Parrella} \\
                  Deadline: \textbf{next meeting}
            \end{flushright}

      \item Write the LCD Display subsection of the Design Brief
            \begin{flushright}
                  Assigned to: \textbf{Gabriel Vella} \\
                  Deadline: \textbf{next meeting}
            \end{flushright}

            % Add more actions as necessary

\end{enumerate}

\end{document}
